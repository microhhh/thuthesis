% !TeX root = ../thuthesis-example.tex

\chapter{绪论}
% \label{cha_intro}

\section{研究背景与意义}
\textbf{细粒度图像识别}是计算机视觉领域的一个非常重要的课题,有着巨大的应用价值。所谓细粒度图像识别,指的是对于相互之间仅仅存在微弱差异的对象进行分类识别,这些对象很可能隶属于同一个大类下的不同子类。典型的细粒度图像识别任务有人脸识别、动植物种类识别和车辆品牌型号识别等。

% \begin{figure}[H] % use float package if you want it here
%   \centering
%   \includegraphics[width=1.0\textwidth]{fig/stanfordcar.jpg}
%   \caption{不同品牌型号的车辆图片}
%   \label{fig:stanfordcar}
% \end{figure}

细粒度图像识别具有技术上的挑战性。除非受过专门的训练、具备专业的知识,对于人类而言,区分相互之间差异如此微小的对象也是极具挑战性的任务。在计算机已经对常规对象识别取得突破进展的今天,细粒度图像识别无疑是等待计算机视觉领域研究者攀登的又一个高峰。对于细粒度图像识别问题的研究将深刻地推动计算机视觉领域的不断发展。在尝试解决这一问题的过程中,涌现出了一系列卓有成效的、颇具创意的方法\cite{zhang2014part, lin2015bilinear,gao2016compact, dubey2018maximum-entropy}。这些方法对计算机视觉领域,乃至对整个机器学习领域,都将带来启发。

细粒度图像识别具有巨大的实际应用价值。近年来,深度学习方兴未艾,深度神经网络的性能不断提高\cite{krizhevsky2012imagenet, simonyan2014very, szegedy2015going, szegedy2016rethinking, he2016deep},计算机视觉领域的研究不断取得新的突破,计算机视觉方法在实际生活中的应用愈发的广泛,其在安防监控、自动驾驶、智能设备等领域均取得了巨大的成功。作为计算机视觉领域的一个重要课题,产业界对于细粒度图像识别同样有着迫切的需求。以机器替代人工能够同时提高效率与质量,并且降低成本,这是计算机视觉已经取得诸多成功应用的原因;而细粒度图像识别方法是以机器替代受过专门训练的专业人员,必然将产生更大的效益。在实际生产中,常常需要对相互之间只存在细微差异的对象进行区分识别。譬如,在电子制造业中,工厂需要对有缺陷的电路板进行识别;在工程机械装备行业中,管理人员需要对施工设备的品牌型号进行识别。

但是,\textbf{标注数据稀缺}问题是限制细粒度图像识别应用的巨大瓶颈,是学术研究和产业应用之间的鸿沟。与一般图像识别模型类似,训练细粒度图像识别模型需要大量的有标注数据。但是,细粒度图像识别的突出问题是,准确识别分类细粒度对象极为困难,标注人员需要经过专门训练,且耗费更大的精力和时间。巨大的人力支出和高昂的成本,造成了细粒度图像数据的严重稀缺。更进一步,数据的稀缺严重制约了细粒度图像识别模型的应用。

幸运的是,迁移学习能够在一定程度上解决机器学习模型训练过程中的数据稀缺问题\cite{pan2010survey}。迁移学习被提出以实现领域间知识的迁移,能够将从某一个任务中学习到的知识应用到另一个任务中去。因为,我们常常面对这样的窘境:我们对某一领域上的任务较为关心,但是缺乏相应的训练数据;与此同时,我们在另一个领域上拥有充足的训练数据。

\textbf{领域适应}是一类重要的迁移学习方法。大部分机器学习和数据挖掘算法基于独立同分布假设,要求模型训练数据和测试数据来自同一个特征空间、共享同一个分布\cite{pan2010survey}。但是,在实际应用中,独立同分布假设常常难以被满足。当独立同分布假设无法满足,即可应用领域适应方法,学习和领域无关的知识,实现知识的跨领域迁移、实现对不符合独立同分布的训练数据的利用。当识别对象为细粒度物体,那么自然地,就需要应用细粒度领域适应方法。


% \begin{figure}[H] % use float package if you want it here
%   \centering
%   \includegraphics[width=1.0\textwidth]{fig/tianyuan.png}
%   \caption{挖掘机图像:停车场(顶部)和施工现场(底部)}
%   \label{fig:tianyuan}
% \end{figure}


\textbf{细粒度领域适应}是细粒度场景下的领域适应,能够完成细粒度类别的跨领域对齐。
于是,利用细粒度领域适应方法,我们能够对现有的、与目标任务存在一定差异的细粒度图像数据加以利用,以此减少数据标注工作量。
例如,所示,挖掘机品牌型号识别是典型的细粒度图像识别问题:停车场和施工现场的挖掘机图像存在明显的差异(背景、工作状态、拍摄视角均存在明显不同);工程机械销售公司已经积累了停车场挖掘机的标注数据,细粒度领域适应方法能够利用现有数据训练图像识别模型,以识别施工现场的挖掘机品牌型号。


\section{研究问题与挑战}
\subsection{研究问题}
本文的研究对象是{\kaishu 细粒度领域适应问题}。为更加清晰地对这一问题进行描述,我们首先将对领域适应和细粒度图像识别进行介绍。在此基础之上,在最后,我们将介绍细粒度领域适应问题。

\textbf{领域适应},是一类典型的迁移学习方法,意在对一个或者多个不同但是相关的领域上的标记数据加以利用,以完成目标领域的任务。%具体定义如下。

注意,这里的领域适应,严谨地讲,是无监督领域适应。
在无监督领域适应问题设定下,在训练过程中,目标领域上的数据的标签是完全不可见的。
本文中的领域适应,在不做强调的情况下,一律默认指的是无监督领域适应。


\begin{definition}
\label{def:da}
给定一个拥有$n_{\mathcal{S}}$个样本的源领域$\mathcal{S}=\{(x,y)\}$,和一个拥有$n_{\mathcal{T}}$个样本的目标领域${\mathcal{T}}=\{(x,?)\}$。源领域和目标领域数据集分别从两个联合分布,$ P(x,y)$和$ Q(x,y)$,采样而得。自然地,这两个分布存在差异,即$P \neq Q$。领域适应的目标在于利用给定数据训练预测模型$\rm M$,使得其在目标领域${\mathcal{T}}=\{(x,?)\}$上的泛化误差$\mathbb{Pr}_{(x,y)\sim{Q}} [{\rm M}(x)≠y]$最小。
\end{definition}


\textbf{细粒度图像识别},是指对于彼此之间仅存在细微差异的图像进行识别的任务。细粒度图像识别问题的显著特点是类间差异小,而类内差异大\cite{Feifei2017fine}。在通常的视觉识别问题中,不同类别的对象之间存在明显的视觉差异;同种类别的对象之间具有很强的视觉共性。但是,在细粒度识别问题中,情况恰恰相反。不同类别的对象之间的差异可能是很细微的,而同种类别的对象可能看上去相差甚远。


\textbf{细粒度领域适应},是指在细粒度场景下的领域适应问题,样本是细粒度对象。
细粒度对象天然具有多层次类别体系。无论是自然界的生物还是日常生活中的人造物品,任何一个细粒度对象均同时属于多个位于不同层次的类别。不同层次的类别之间存在隶属关系。
例如,所有生物均可以按照界、门、纲、目、科、属、种这七个分类单位进行分类;工业制品同样可以按照厂家、产品线、型号、年份等若干层次进行分类。
为方便表述,我们将较大的、粗略的、位于高层次的类别称为粗粒度类别;将较小的、细致的、位于低层次的类别称为细粒度类别。位于高层次的粗粒度类别可以包含位于低层次的细粒度类别。
细粒度领域适应的定义如\ref{def:fine}。

\begin{definition}
\label{def:fine}
给定一个拥有$n_{\mathcal{S}}$个细粒度样本的源领域$\mathcal{S} = \{({x},{y}_f,{y}_c^k|_{k=1}^K)\}$,和一个拥有$n_{\mathcal{T}}$个细粒度样本的目标领域${{\mathcal{T}}} = \{({x}, \textrm{?}, \textrm{?} )\}$。源领域和目标领域数据集分别从两个联合分布,${P}(x,{y}_f,{y}_c^k|_{k=1}^K)$和${ Q}(x,{y}_f,{y}_c^k|_{k=1}^K)$,采样而得。自然地,这两个分布存在差异,即$P \neq Q$。领域适应的目标在于利用给定数据训练预测模型$\rm M$,使得其在目标领域${\mathcal{T}}=\{(x,?,?)\}$上的泛化误差$\mathbb{Pr}_{(x,{y}_f,{y}_c^k|_{k=1}^K)\sim{Q}} [{\rm M}(x)≠y_f]$最小。
\end{definition}



\subsection{面临挑战}
\textbf{细粒度图像识别}具有类间差异相对小、类内差异相对大这一显著特点。相邻类别很可能视觉相似度很高,隶属于相邻类别的个体可能共享了诸多特征。与此同时,隶属于同种类别的不同个体之间可能存在明显视觉差异。
\textbf{领域适应}面临的核心问题是较大的领域间差异,这构成了知识跨领域迁移的主要障碍。


% \begin{figure}[H] % use float package if you want it here
%   \centering
%   \includegraphics[width=0.75\textwidth]{fig/fine-grained_DA_1.pdf}
%   \caption{细粒度领域适应问题。细粒度领域适应问题最为突出的特点是:较小的类间差异、较大的类内差异、以及较大的领域间差异耦合在一起。从左到右,每列分别是:红翼乌鸫,黄头乌鸫,北美食米鸟,褐旋木雀。}
%   \label{fig:problem}
% \end{figure}


\textbf{细粒度领域适应}兼具细粒度图像识别和领域适应这两个问题的特点。一方面,分类对象的类间差异相对小,而类内差异相对大。另一方面,源领域和目标领域之间存在巨大的差异。
{\kaishu 较小的类间差异},{\kaishu 较大的类内差异}和{\kaishu 较大的领域间差异}耦合在一起,构成了细粒度领域适应问题的突出特点。
这三者交错叠加、相互影响、共同作用,急剧提升了细粒度领域适应问题的挑战性。深度网络擅长处理类间差异大、类内差异小的样本的分类任务。面对细粒度对象,深度网络并不能够捕捉到有足够区分度的细节信息,无法提取到类间区分度大的、类内聚集度好的特征。
细粒度对象的类别边界非常模糊,互相嵌入现象严重,较一般对象的类别边界远为微妙。
失去高质量特征作为基础,导致进行跨领域类别对齐时,相邻的类别很容易被错误匹配到一起,严重影响领域适应的效果。



图是对细粒度领域适应问题特有的较小的类间差异、较大的类内差异和较大的领域间差异这三个特点的形象展示。
较小的类间差异:不同物种彼此间视觉差异可能较小,尤其以黄头乌鸫(第二列)和北美食米鸟(第三列)最为明显,这两种鸟类均由黑色和黄色构成主色调;
较大的类内差异:同一物种不同个体(同列不同行)之间差异可能较大,拍摄背景不同的图片存在明显视觉差异;
较大的领域间差异:分别位于图中上半部分和下半部分的源领域和目标领域图片存在巨大的视觉差异,二者分别为手工图画和真实照片。




\section{研究现状}
本节是对细粒度迁移学习相关工作的概要总结。详细的文献综述见第章。

本文的研究对象是{\kaishu 细粒度无监督领域适应问题}。
但是,作为一个全新的问题,{ 目前学术界对细粒度无监督领域适应的研究才刚刚展开}。
领域适应是一类重要的迁移学习方法。于是,我们对细粒度迁移学习方法进行总结,现有工作可以分为两类:细粒度半监督领域适应和细粒度微调。
遗憾的是,仅有极少的相关工作。

\textbf{细粒度半监督领域适应1。}
文献\cite{Feifei2017fine}是最早的关于细粒度领域适应问题的研究。
利用领域和任务同时迁移方法(Simultaneous Deep Transfer Across Domains and Tasks,STDT)\cite{tzeng2015simultaneous},Gebru等 \cite{Feifei2017fine}在有类别标记信息的网络数据集上训练模型,之后将训练完毕的模型应用到户外车辆品牌型号识别任务中。为充分挖掘细粒度对象特有的属性信息,Gebru等提出了一种专门针对细粒度领域适应问题的半监督适应损失函数。但是,应用半监督适应损失函数的前提是目标领域上的数据是部分有标签的。所以说,文献\cite{Feifei2017fine}是关于细粒度半监督领域适应问题的研究。

\textbf{细粒度半监督领域适应2。}
文献 \cite{xu2018webly}提出了一种特别的设计,除标准图像级别的标签之外,还利用了对象的边界框和特殊点标记等强监督信息,实现了将尽可能多的知识从现有的强监督数据集迁移到弱监督网络图像数据集。
同样地,应用这一方法的前提是目标领域部分数据是有标签的。

\textbf{细粒度微调。}
文献 \cite{cui2018large} 首先利用(Earth Mover's Distance,EMD)对源领域和目标领域各个类别之间的距离进行度量;然后从源领域中筛选出最具迁移价值的类别;最后,在源领域上对模型进行预训练,在目标领域上微调(Fine-tune)预训练模型,完成知识的迁移。需要注意的是,在这里,源领域和目标领域的数据都是有标记的,源领域数据集的规模远远大于目标领域,二者并不共享类别空间。


上述方法均取得了令人鼓舞的效果。但是,他们所研究的{\kaishu 问题的设定与本文的并不一致},如=所示。
我们的方法不需要属性、边框或部位标记等额外信息,而是依赖于在细粒度任务中更容易获得的多层次类别标签,并且在训练中不要求目标领域数据有任何标记信息。
{\kaishu 本文的工作是第一个仅依赖源领域的图像级标签的细粒度无监督领域适应方法 \cite{wangprogressive}。}


% \begin{table*}[t]
% % \begin{small}
% \caption[模板文件]{不同细粒度迁移学习方法所依赖的标注信息的对比  ($-$: 部分地)。}
% % \vspace{-7.5pt}
% \addtolength{\tabcolsep}{7.25pt}
% \centering
% \label{table:comparison} 
% \begin{tabular}{l|cc|ccc}
% \toprule[1.5pt]
% \multirow{2}*{方法} & \multicolumn{2}{c|}{需要图像级标签?} & \multicolumn{3}{c}{需要额外标注?}\\
% & 源领域 & 目标领域 & 属性  & 边框 & 部位 \\
% \midrule
% Gebru等 \cite{Feifei2017fine}& $\surd$&  $-$& $\surd$& $\times$ &$\times$   \\
% Xu等 \cite{xu2018webly} &$\surd$  &$-$    &$\times$ & $\surd$ & $\surd$ \\
% Cui等 \cite{cui2018large} & $\surd$ &$\surd$    &$\times$ &$\times$  &$\times$   \\
% 我们的 & $\surd$ (多层次的) &  $\times$  &$\times$ &  $\times$& $\times$  \\
% \bottomrule[1.5pt]
% \end{tabular}
% % \end{small}
% \end{table*}






\section{研究内容与贡献}
本文主要研究内容为细粒度领域适应问题。细粒度场景下的领域适应问题有其不同于一般领域适应问题的特殊性。
如上文所述,较小的类间差异、较大的类内差异、以及较大的领域间差异耦合在一起,构成了细粒度领域适应问题的主要挑战。
经典的领域适应方法通过拉近源领域和目标领域的分布,来实现跨领域的知识迁移。
但是,在细粒度场景下,经典的领域适应方法难以奏效,新引入的较小的类间差异和较大的类内差异使得情况急剧复杂。
深度网络难以提取高质量特征,不同细粒度类别之间的界限变得异常模糊,跨领域地将相应细粒度类别进行正确地对齐变得十分困难。


\textbf{本文提出渐进对抗网络。}
渐进对抗网络是第一个仅依赖图像级类别标记信息的细粒度无监督领域适应方法 \cite{wangprogressive}。
细粒度对象特有多层次类别体系。无论是自然界的生物还是生活中的人造物品,这一点均成立。这为应用渐进学习思想提供了条件。
高层次的类别可以称为粗粒度类别;相应地,低层次的类别可以称为细粒度类别。相对容易的粗粒度领域适应可以作为进行困难的细粒度领域适应的基础。
渐进学习,或者说课程学习\cite{bengio2009curriculum},是从容易到困难的学习过程,符合人类的认知过程。
将渐进学习和对抗学习结合,渐进对抗网络充分挖掘利用了细粒度对象特有的多层次类别体系,使得相应类别的跨领域对齐逐渐从粗粒度类别过渡到细粒度类别、逐渐从容易过渡到困难。
本文提出的渐进对抗网络具有以下贡献:

1. 提出了渐进粒度学习,将渐进学习应用到类别分类中。为实现在源领域上的类别预测器训练从粗粒度逐渐过渡到细粒度、从容易逐渐过渡到困难,本文提出粗细粒度混合损失函数,实现了将分类器训练的监督信息从粗粒度逐渐过渡到细粒度,成功构建了动态变化的多粒度特征。这为渐进的跨领域对齐打下了基础。
 
2. 提出了渐进对抗学习,将渐进学习应用到特征分布的跨领域对齐中,实现了与对抗学习的结合。
针对细粒度图像特征相互之间差异细微的特点,本文提出了专门针对细粒度场景的类别预测分布-特征嵌入方法。
借助这一方法,类别预测器的预测分布被嵌入到样本特征中,共同参与领域对抗,实现了跨领域类别匹配从粗粒度逐渐过渡到细粒度、从困难逐渐过渡到容易。


\textbf{本文构建了两个全新的数据集,并在其上取得了最佳效果。}
细粒度领域适应是一个全新的问题,学术界对此的研究才刚刚展开,尚且缺乏相应的基础设施。
统一的评价标准(数据集)是推动一个领域研究不断进步的基础。
由于缺乏公开数据集,利用细粒度图像识别公开数据集和网络爬取到的图像数据,本文构建并公开了两个专门用于评估细粒度领域适应方法的数据集,CUB-Paintings和Birds-31。
在这两个数据集上,本文对渐进对抗网络和业界最先进的方法的各项性能进行了对比。在各个方面,本文提出的渐进对抗网络均取得了最佳表现。

\textbf{本文所提出的方法实现了系统集成。}
近年来涌现了许多深度学习框架,比如Caffe\cite{jia2014caffe}、Tensorflow\cite{abadi2016tensorflow}、MXNet\cite{chen2015mxnet}、Pytorch\cite{paszke2019pytorch}。经过社区的不断努力、版本的不断迭代,越来越多的高层次封装已经被包含在上述深度学习框架中。
但是,对基础较为薄弱的用户而言,这些深度学习框架依然不够简单易用,应用深度学习方法的门槛依然过高。
Xlearn-DA是大数据系统软件国家工程实验室所设计开发的领域适应学习系统。借助Xlearn-DA,用户能够方便地完成领域适应模型的训练和部署。
本文提出的渐进对抗网络被集成到Xlearn-DA中,丰富了Xlearn-DA的算法库。
% 在监控视频车辆品牌型号识别任务中,利用有类别标记的网络车辆图片,渐进对抗网络成功训练模型,对监控视频中出现的车辆的品牌型号进行了分类识别,实现了方法的实际应用。
在监控视频车辆品牌型号识别任务中,利用有类别标记的网络车辆图片,本文提出的渐进对抗网络成功地训练模型,对监控视频中出现的车辆的品牌型号进行了分类识别,效果远超现有方法。

\section{文章结构安排}
本文共计包含5个章节,主要内容安排如下:

第1章为绪论部分,主要介绍了细粒度领域适应的研究背景及其意义,定义了细粒度领域适应这一全新问题,并概括了本文的研究内容和主要贡献。
另外,本章还简要概括了细粒度迁移学习的研究现状,对比了现有工作问题设定上的差异。

第2章为相关工作综述,从领域适应、细粒度图像识别和细粒度迁移学习这三方面着手进行文献调研和综述,囊括了学术界在这三个方面最新的研究成果。

第3章提出了渐进对抗网络。这是第一个仅仅依赖图像级标签的细粒度无监督领域适应方法。融合渐进学习和对抗学习思想,渐进对抗网络充分挖掘细粒度对象特有的多层次类别体系,极大提高了跨领域细粒度类别对齐的准确率。渐进对抗网络可以被划分为渐进粒度学习和渐进对抗学习两个部分。此二者密切协作,实现了类别跨领域对齐从粗粒度到细粒度、从容易到困难的渐进过渡。

第4章为实验分析与方法应用。由于缺乏公开数据集,本文构建了两个专门用于评价细粒度领域适应方法的图像数据集,CUB-Paintings和Birds-31。随后,在这两个数据集上,通过一系列实验对模型各项性能进行了评价。本文提出的渐进对抗网络超过学术界最新成果,在各项对比中均取得了最佳表现。最后,渐进对抗网络被集成到大数据系统软件国家工程实验室的领域适应学习系统Xlearn-DA中,并应用到了监控视频车辆品牌型号识别任务中。

第5章对本文进行了总结,并对未来的可能的研究方向进行了展望。


% \begin{itemize}
%   \item 问题的提出:要清晰地阐述所要研究的问题“是什么”。
%     \footnote{选题时切记要有“问题意识”,不要选不是问题的问题来研究。}
%   \item 选题背景及意义:论述清楚为什么选择这个题目来研究,即阐述该研究对学科发展的贡献、对国计民生的理论与现实意义等。
%   \item 文献综述:对本研究主题范围内的文献进行详尽的综合述评,“述”的同时一定要有“评”,指出现有研究状态,仍存在哪些尚待解决的问题,讲出自己的研究有哪些探索性内容。
%   \item 研究方法:讲清论文所使用的学术研究方法。
%   \item 论文结构安排:介绍本论文的写作结构安排。
% \end{itemize}



